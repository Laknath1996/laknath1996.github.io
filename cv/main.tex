%%%%%%%%%%%%%%%%%%%%%%%%%%%%%%%%%%%%%%%%%
% "ModernCV" CV and Cover Letter
% LaTeX Template
% Version 1.1 (9/12/12)
%
% This template has been downloaded from:
% http://www.LaTeXTemplates.com
%
% Original author:
% Xavier Danaux (xdanaux@gmail.com)
%
% License:
% CC BY-NC-SA 3.0 (http://creativecommons.org/licenses/by-nc-sa/3.0/)
%
% Important note:
% This template requires the moderncv.cls and .sty files to be in the same 
% directory as this .tex file. These files provide the resume style and themes 
% used for structuring the document.
%
%%%%%%%%%%%%%%%%%%%%%%%%%%%%%%%%%%%%%%%%%

%----------------------------------------------------------------------------------------
%	PACKAGES AND OTHER DOCUMENT CONFIGURATIONS
%----------------------------------------------------------------------------------------

\documentclass[10pt,a4paper,sans]{moderncv} % Font sizes: 10, 11, or 12; paper sizes: a4paper, letterpaper, a5paper, legalpaper, executivepaper or landscape; font families: sans or roman
\usepackage{standalone}
\moderncvstyle{classic} % CV theme - options include: 'casual' (default), 'classic', 'oldstyle' and 'banking'
\moderncvcolor{blue} % CV color - options include: 'blue' (default), 'orange', 'green', 'red', 'purple', 'grey' and 'black'

\usepackage{lipsum} % Used for inserting dummy 'Lorem ipsum' text into the template

\usepackage[scale=0.85]{geometry} % Reduce document margins
%\setlength{\hintscolumnwidth}{3cm} % Uncomment to change the width of the dates column
%\setlength{\makecvtitlenamewidth}{10cm} % For the 'classic' style, uncomment to adjust the width of the space allocated to your name

%\usepackage[utf8]{inputenc}

%\usepackage{booktabs}
\usepackage{fontawesome}
\usepackage{marvosym} % For cool symbols.
% \usepackage{hyperref}
\usepackage{footmisc}



%----------------------------------------------------------------------------------------
%	NAME AND CONTACT INFORMATION SECTION
%----------------------------------------------------------------------------------------

\firstname{Ashwin} % Your first name
\familyname{De Silva} % Your last name

% All information in this block is optional, comment out any lines you don't need
% \title{Curriculum Vitae}
\address{Clark Hall 317}{Baltimore MD 21218}
% \mobile{(+91) 0000000, (+91) 0000000}


%\fax{(000) 111 1113}
 
%\social{github}{stefano-bragaglia}
\email{ldesilv2@jhu.edu} 



\homepage{https://laknath1996.github.io}{personal website}

% social link \faGithub, \faSkype, \faLinkedin,\faStackExchange, and \faStackOverflow
\extrainfo{
    \faGraduationCap\href{https://scholar.google.com/citations?user=xqhwEGIAAAAJ}{ Google Scholar} \quad
    \faGithub\href{https://github.com/Laknath1996}{ Github} \quad
    \faLinkedin\href{https://www.linkedin.com/in/ashwin-de-silva-6852b14b/}{ Linkedin} \quad
    % \faTwitter\href{https://twitter.com/AshwindeSilva1}{ Twitter} \quad
    }

%\social[linkedin][www.linkedin.com]{name}
% The first argument is %the url for the clickable link, the second argument is the url displayed in the %template - this allows special characters to be displayed such as the tilde in this %example

% \photo[70pt][0.3pt]{picture} % The first bracket is the picture height, the second is %the thickness of the frame around the picture (0pt for no frame)
%\quote{Not Attention, Patience is all we need.}

%----------------------------------------------------------------------------------------

\newcommand{\cvdoublecolumn}[2]{%
  \cvitem[.75em]{}{%
    \begin{minipage}[t]{\listdoubleitemcolumnwidth}#1\end{minipage}%
    \hfill%
    \begin{minipage}[t]{\listdoubleitemcolumnwidth}#2\end{minipage}%
    }%
}



\usepackage{multibbl}
\newcommand\Colorhref[3][orange]{\href{#2}{\small\color{#1}#3}}


% \newcommand{\cvreference}[7]{%
%     \textbf{#1}\newline% Name
%     \ifthenelse{\equal{#2}{}}{}{\addresssymbol~#2\newline}%
%     \ifthenelse{\equal{#3}{}}{}{#3\newline}%
%     \ifthenelse{\equal{#4}{}}{}{#4\newline}%
%     \ifthenelse{\equal{#5}{}}{}{#5\newline}%
%     \ifthenelse{\equal{#6}{}}{}{\emailsymbol~\texttt{#6}\newline}%
%     \ifthenelse{\equal{#7}{}}{}{\phonesymbol~#7}}

\begin{document}

\makecvtitle % Print the CV title
% About Me
% I am a PhD candidate in the Department of Biomedical Engineering at Johns Hopkins University 
% where I am advised by Prof. Joshua Vogelstein, Prof. Pratik Chaudhari (UPenn), and Prof. Carey E. Priebe. 
% I work on \{machine, deep\} learning, with an aspiration of bridging the gap between 
% machine and natural intelligence. My doctoral research focuses on continual learning, 
% out-of-distribution (OOD) generalization, and learning under non-stationary 
% distributions, with
% applications in large language models, computer vision, 
% and biomedical data science.

Ph.D. candidate in Biomedical Engineering with expertise in statistical machine learning, deep learning and signal processing, focused on developing theory and robust methods for prospective learning, continual learning and out-of-distribution generalization. Passionate about bridging artificial and natural intelligence. My work spans applications in large language models, biomedical data, and computer vision.

%----------------------------------------------------------------------------------------
%	EDUCATION SECTION
%----------------------------------------------------------------------------------------

\section{Education}
\cventry{2021--present}{Ph.D., Biomedical Engineering}{Johns Hopkins University}{MD}{USA}
{\textit{Highlighted Courses}: Probability Theory, Statistical Theory, High-Dimensional Approximation, Machine Learning, Optimal Transport, Probabilistic Models of Visual Cortex, Neuroscience and Cognition, Computational Molecular Medicine, Compressed Sensing \& Sparse Recovery}
\cvitem{CGPA :}{3.97/4.00}

\cventry{2021--2024}{M.S.E., Applied Mathematics and Statistics}{Johns Hopkins University}{MD}{USA}
{\textit{Focus Area}: Statistics and Statistical Learning}
\cvitem{CGPA :}{3.97/4.00}

\cventry{2016--2020 :}{B.Sc., Biomedical Engineering}{University of Moratuwa}{Sri Lanka}{}{Class Rank : 1 out of 117, Faculty Rank : 1 out of 948, Included in Dean's Honors List every semester. \\ \textit{Highlighted Courses}: Real Analysis, Calculus, Differential Equations, Linear Algebra, Signals and Systems, Machine Vision, Digital Signal Processing, Data Structures \& Algorithms}{}
\cvitem{CGPA :}{4.09/4.20 (First Class Honors)}

% \let\thefootnote\relax\footnotetext{For exhaustive lists of courses and research works, please visit: \textcolor{blue}{\href{https://laknath1996.github.io/vitae/}{list of courses}}, \textcolor{blue}{\href{https://laknath1996.github.io/research/}{list of research works}} }

%----------------------------------------------------------------------------------------
%	WORK EXPERIENCE SECTION
%----------------------------------------------------------------------------------------

\section{Research Experience}
\cventry{Summer 2025}{Amazon (AWS AI), Santa Clara, CA}{Applied Scientist Intern}{}{}{Developed self-adaptive context recomposition methods for multi-turn tool-using agents that reduced the input token utilization by 50\% while maintaining the task success rate across mathematical reasoning and software engineering benchmarks}
\cventry{Summer 2024}{Amazon (AWS AI), Santa Clara, CA}{Applied Scientist Intern}{}{}{Developed large language models (LLMs) based on state-space models (SSMs) to perform code generation and constrained generation}
\cventry{2021-Present}{Johns Hopkins University, Baltimore, MD}{Graduate Research Assistant}{}{}{Building theory and methods for learning from non-stationary and out-of-distribution data, with applications in large language models, computer vision, and biomedical data science}
\cventry{2019-2021}{University of Moratuwa, Sri Lanka}{Junior Lecturer}{}{}{Developed deep learning models for various problems including retinal vascular segmentation, surface EMG based hand gesture recognition, human pose estimation, and phase unwrapping}
\cventry{2018}{Center for Advanced Imaging, University of Queensland, Australia}{Research Intern}{}{}{Developed convolutional neural networks (CNNs) to solve the inverse problem of phase unwrapping}
\cventry{2017,2018}{The Florey Institute of Neuroscience, University of Melbourne, Australia}{Research Intern}{}{}{Developed machine learning and signal processing algorithms for charaterizing epileptogenic mutations based on multi-electrode array (MEA) recordings acquired from in-vitro neuronal networks. }

%----------------------------------------------------------------------------------------
%	PUBLICATION SECTION
%----------------------------------------------------------------------------------------

\section{Publications}

\newbibliography{selected_pubs}
\nocite{selected_pubs}{*}
\bibliographystyle{selected_pubs}{plainyrrev}
\bibliography{selected_pubs}{selected_pubs}
{\large \textsc{Refereed Conference Publications}}

% \subsection{In Conference Proceedings}
% \newbibliography{conference}
% \nocite{conference}{*}
% \bibliographystyle{conference}{plainyrrev}
% \bibliography{conference}{conference}
% {\large \textsc{Refereed Conference Publications}}

% \subsection{Journal Articles}
% \newbibliography{journals}
% \nocite{journals}{*}
% \bibliographystyle{journals}{plainyrrev}
% \bibliography{journals}{journals}
% {\large \textsc{Refereed Conference Publications}}

\subsection{Preprints}
\newbibliography{preprints}
\nocite{preprints}{*}
\bibliographystyle{preprints}{plainyrrev}
\bibliography{preprints}{preprints}
{\large \textsc{Refereed Conference Publications}}

% \subsection{Workshop Papers}
% \newbibliography{workshop_papers}
% \nocite{workshop_papers}{*}
% \bibliographystyle{workshop_papers}{plainyrrev}
% \bibliography{workshop_papers}{workshop_papers}
% {\large \textsc{Refereed Conference Publications}}

% \subsection{Theses}
% \newbibliography{theses}
% \nocite{theses}{*}
% \bibliographystyle{theses}{plainyrrev}
% \bibliography{theses}{theses}
% {\large \textsc{Refereed Conference Publications}}

%----------------------------------------------------------------------------------------
%	RESEARCH EXPERIENCE SECTION
%----------------------------------------------------------------------------------------

% \section{Research Experience}
% \subsection{Johns Hopkins University, MD, USA}
% \subsection{University of Moratuwa, Sri Lanka}
% \subsection{Florey Institute of Neuroscience and Mental Health, University of Melbourne, Australia}
% \subsection{Center for Advanced Imaging, University of Queensland, Australia}

% \section{Research Experience}
% \subsection{Indian Institute of Technology, ABC}
% \cventry{June,2019 -- present}{\textit{Identifying Protein-protein Interaction from Biomedical text}}{}{}{}
% {Developing a deep multi-modal architecture for accurately predicting protein interaction information from biomedical text. 
% }
% \cvitem{Advisor :}{\textbf{Dr. abc xyz}, \textit{Associate Professor, Department of Computer Science \& Engineering}, IIT abc ({\Colorhref{https://www.personal_webpage.com/} {\textit{Personal Web-page}}})}

% \cventry{July,2018 -- present}{\textit{Developing Deep Multi-modal Architecture for Biomedical Problems}}{}{}{}
% {Analyzing different modalities of genes like gene expression profiles, protein 3D structure, underlying amino acid sequence using popular deep learning models to obtain deeper insight into the underlying biological system. 
% }
% \cvitem{Advisor :}{\textbf{Dr. abc xyz}, \textit{Associate Professor, Department of Computer Science \& Engineering}, IIT abc ({\Colorhref{https://www.personal_webpage.com/} {\textit{Personal Web-page}}})}

% \subsection{Indian Institute of Technology, XYZ}
% \cventry{January,2015 -- Dec,2015}{\textit{Design and Synthesis of Reversible Multi-dimentional Nearest-Neighbour(NN) Quantum Circuit}}{}{}{}{Proposed an approach for designing and physically implementing of the multi-dimensional quantum circuits maintaining nearest-neighbor complacency that use minimal number of SWAP gates.}
% \cvitem{Advisor :}{\textbf{Dr. abc xyz}, \textit{Associate Professor, Department of Computer Science \& Engineering}, IIT abc ({\Colorhref{https://www.personal_webpage.com/} {\textit{Personal Web-page}}})}


% \cventry{2012 -- 2013}{\textit{Text Document Clustering with Semantic Similarity through Wordnet}}{}{}{}{Improvement of the text document clustering task over conventional methods by introducing WORDNET and some better clustering algorithms.}
% \cvitem{Advisor :}{\textbf{Dr. abc xyz}, \textit{Associate Professor, Department of Computer Science \& Engineering}, IIT abc ({\Colorhref{https://www.personal_webpage.com/} {\textit{Personal Web-page}}})}

%----------------------------------------------------------------------------------------
%	Fellowships \& Awards
%----------------------------------------------------------------------------------------

% \section{Fellowships \& Awards}

% \cvitem{2016 --present}{\textit{\textbf{Visvesvaraya Fellowship}} of Ministry of Electronics and Information Technology (MeitY), Government of India, as a PhD research scholar in Indian Institute of Technology Patna.}
% \cvitem{2019}{Receipt of \textit{\textbf{Visvesvaraya Travel Grant}} to attend a international conference \textbf{\textit{IEEE Congress on Evolutionary Computation, 2019}} in Wellington, New Zealand.}
% \cvitem{2018}{Recipient of \textit{\textbf{SciGenome Research Foundation (SGRF) GYAN Scholarship}} to participate \textbf{\textit{Nextgen Genomics, Biology, Bioinformatics and Technologies-2018}} meeting at Jaipur India from $30^{th}$ September to $2^{nd}$ October 2018. }
% \cvitem{2015}{Awarded under \textit{\textbf{Students Reward Programme}} at the Annual General Meeting of \textbf{Global Alumni Association of Bengal Engineering and Science University(GAABESU).}}

%----------------------------------------------------------------------------------------
%	Academic achievements
%----------------------------------------------------------------------------------------

\section{Academic Achievements \& Recognitions }

\cvitem{2026}{\textbf{Best Student Paper Award} at the 18th Annual Conference on Artificial General Intelligence (AGI-25), Reykjavík, Iceland}
\cvitem{2025}{\textbf{Member, Alpha Eta Mu Beta (AEMB) – National Biomedical Engineering Honor Society} Invited for membership in recognition of academic excellence, ranking in the top third of the Biomedical Engineering PhD class.}
\cvitem{2024}{\textbf{MINDS Fellowship} selected as a fellow of the Mathematical Institute of Data Science, Johns Hopkins University}
\cvitem{2023}{\textbf{Johns Hopkins School of Medicine Student Spotlight} for research and academic accomplishments}
\cvitem{2022}{\textbf{Best Paper Award} at the ECCV 2022 Workshop on Out-of-distribution Generalization in Computer Vision, Tel Aviv, Israel}
\cvitem{2021}{\textbf{2nd Runners-up of the IEEE Video and Image Processing Cup} awarded at the International Conference on Image Processing (ICIP) 2021, Anchorage, Alaska, USA}
\cvitem{2020}{\textbf{Prof. Pathuwathawithana Memorial Prize} for attaining the \emph{highest} GPA at the Faculty of Engineering, University of Moratuwa, Sri Lanka}
\cvitem{2020}{\textbf{Gold Medal sponsored by Technomedics International Pvt Ltd} for the \emph{highest} overall academic performance in the Biomedical Engineering Stream (University of Moratuwa)}
\cvitem{2020}{\textbf{National Finalists at the Migara Ranatunga Awards} awarded by Institution of Engineers, Sri Lanka (IESL) for the \emph{best} performance in the research internship}
\cvitem{2019}{\textbf{World Finalists at the IEEE ComSoc Student Competition} ranked among \emph{the top 15 in the world}, received an Honorable Mention}
\cvitem{2019}{\textbf{Merit Award at SLAAS Awards} awarded by Sri Lanka Association for the Advancement of Science (SLAAS) for the \emph{best undergraduate} project in the country}
\cvitem{2019}{\textbf{National Finalists at the Sri Lankan IoT Challenge} ranked among \emph{the top 10 in the country}, received an Honorable Mention}
\cvitem{2019}{\textbf{Runners-Up at the the National Inter-University Statistics Quiz Competition} Organized by University of Sri Jayawardenapura, Sri Lanka}
\cvitem{2016}{\textbf{Dialog Merit Scholarship for Engineering Undergraduates} awarded by Dialog Axiata PLC for the students who excelled at the university entrance examinations at the national level (country rank: 10 out of $\sim$ 35,000 in the physical science stream)}
\cvitem{2016}{\textbf{Mahapola Merit Scholarship for Engineering Undergraduates} awarded by the Government of Sri Lanka for the students who excelled at the university entrance examinations}
\cvitem{2015}{\textbf{Darrel Medal} awarded by Richmond College, Sri Lanka for the most outstanding advanced level student.}

%----------------------------------------------------------------------------------------
%	TEACHING SECTION
%----------------------------------------------------------------------------------------
\section{Selected Teaching Experience}
\subsection{Teaching Assistant}
\cventry{2025 Fall:}{EN.580.697 Biomedical Data Design}{}{JHU}{USA.}{}

\subsection{Junior Lecturer}
\cventry{2021 Fall:}{EN 1060 Signals and Systems}{}{UoM}{Sri Lanka}{}
\cventry{2020 Fall:}{EN 2030 Laboratory Practice II}{}{UoM}{Sri Lanka}{}
\cventry{2020 Spring:}{EN 3030 Circuits and Systems Design}{}{UoM}{Sri Lanka}{}
\cventry{2020 Spring:}{BM 4111 Medical Electronics and Instrumentation}{}{UoM}{Sri Lanka}{}
\cventry{2020 Fall:}{BM 2101 Analysis of Physiological Systems}{}{UoM}{Sri Lanka}{}
\cventry{2020 Fall:}{BM 2011 Human Anatomy and Physiology}{}{UoM}{Sri Lanka}{}
\cventry{2019 Fall:}{EN 1093 Laboratory Practice I}{}{UoM}{Sri Lanka}{}
% \cventry{2018 Spring:}{DE 2410 Astronomy and Cosmology}{}{UoM}{Sri Lanka}{}

\subsection{Visiting Lecturer}
\cventry{2020 Spring:}{Workshop on MATLAB for signal/image processing, communication systems, and electronics}{}{Institute of Engineering Technology}{Sri Lanka}{}

%----------------------------------------------------------------------------------------
%	Reviewing activities
%----------------------------------------------------------------------------------------

\section{Reviewing Activities}
\cvitem{}{\textbf{Reviewer} NeurIPS 2025, Conference on Language Models (COLM) 2025, Moratuwa Engineering Research Conference (MERCon) 2024}


%----------------------------------------------------------------------------------------
%	TECHNICAL SKILLS SECTION
%----------------------------------------------------------------------------------------

\section{Technical skills}

\cvitem{}{\textit{Programming Languages}: Python, MATLAB, C/C++, Verilog HDL, \LaTeX}
\cvitem{}{\textit{Frameworks}: PyTorch, PyG (PyTorch Geometric), Tensoflow, Keras, scikit-learn, ITK/VTK}
\cvitem{}{\textit{Software}: Quartus, Multisim, AutoCAD, Altium, Solidworks}
\cvitem{}{\textit{Hardware}: STM32 Family, Atmel AVR, Altera DE2, Raspberry Pi, Arduino}
% \cvitem{Web Technologies}{HTML 5, PHP, JSP, Javascript}
% \cvitem{Database}{SQL, MySQL, Apache, Neo4j}

%----------------------------------------------------------------------------------------
%	SELECTED TALKS
%----------------------------------------------------------------------------------------

\section{Selected Talks}

\cvitem{Dec. 2024}{\textit{Prospective Learning}, Center for Imaging Science (CIS) Retreat, Johns Hopkins University, MD, USA}
\cvitem{Sep. 2022}{Critique on \textit{Invertible Neural Networks for Graph Predictions}, Theorinet Retreat, Simons Foundation, NY, USA}
\cvitem{Oct. 2022}{\textit{The Value of Out-of-distribution Data}, ECCV 2022 workshop on Out-of-distribution Generalization in Computer Vision, Tel Aviv, Israel}
% \cvitem{Jun. 2021}{\textit{A Joint Convolutional and Spatial Quad-Directional LSTM Network for Phase Unwrapping}, 46th International Conference on Acoustics, Speech, and Signal Processing (ICASSP), Toronto, Canada}
% \cvitem{Jun. 2020}{\textit{Real-time hand gesture recognition using temporal muscle activation maps of
% multi-channel sEMG signals}, 45th International Conference on Acoustics, Speech, and Signal Processing (ICASSP), Barcelona, Spain}

%----------------------------------------------------------------------------------------
%	SELECTED POSTER PRESENTATIONS
%----------------------------------------------------------------------------------------

\section{Selected Poster Presentations}

\cvitem{Jul. 2025}{\textit{Prospective Learning: Learning for a Dynamic Future}, NeuroAI 2025, Allen Institute, Seattle, WA, USA}
\cvitem{Apr. 2025}{\textit{Prospective Learning: Learning for a Dynamic Future}, Johns Hopkins Data Science and AI Institute Spring 2025 Symposium, Baltimore, MD, USA}
\cvitem{Dec. 2024}{\textit{Prospective Learning: Learning for a Dynamic Future}, NeurIPS 2024 workshop on NeurIPS Workshop on NeuroAI: Fusing Neuroscience and AI for Intelligent Solutions, Vancouver, Canada}
\cvitem{Sept. 2024}{\textit{Prospective Learning}, Mathematical and Scientific Foundations of Deep Learning Annual Meeting (MoDL), Simons Foundation, NY, USA}
\cvitem{Dec. 2022}{\textit{The Value of Out-of-distribution Data}, NeurIPS 2022 Workshop on distribution shifts (DistShift), New Orleans, LA, USA}
\cvitem{Apr. 2022}{\textit{Kernel Density Networks}, From Neuroscience to Artificially Intelligent Systems (NAISys), Cold Spring Harbor Laboratory, NY, USA}
% \cvitem{Jun. 2021}{\textit{A Joint Convolutional and Spatial Quad-Directional LSTM Network for Phase Unwrapping}, 46th International Conference on Acoustics, Speech, and Signal Processing (ICASSP), Toronto, Canada}

%----------------------------------------------------------------------------------------
%	PROFESSIONAL ACTIVITIES
%----------------------------------------------------------------------------------------

% \section{Professional Activities}
% \subsection{Organizing}
% \cvitem{2021}{\textit{IEEE EMBS International Student Conference}, Moratuwa, Sri Lanka}
% \cvitem{2018}{\textit{Workshop on Brain Computer Interfaces}, MerCon 2018, Moratuwa, Sri Lanka}
% \cvitem{2017}{\textit{TechMedImpact Forum}, Sri Lanka}

% \subsection{Reviewing (Conferences)}
% \cvitem{2021}{\textit{IEEE EMBS International Student Conference}, Moratuwa, Sri Lanka}
% \cvitem{2021}{\textit{MerCon 2021}, Moratuwa, Sri Lanka}

%----------------------------------------------------------------------------------------
%	SERVICES AND LEADERSHIP
%----------------------------------------------------------------------------------------

\section{Services and Leadership}
\cventry{2018-Present}{Richmond to University (R2U) Foundation}
{Co-Founder}{}{}{
  \begin{itemize}
    \item An alumni-run organization aimed at organizing career guidance programs for the students of Richmond College, Sri Lanka
  \end{itemize}
}

\cventry{2016-2020}{IEEE Engineering in Medicine and Biology Student Branch Chapter, University of Moratuwa}
{Chairperson 2019/20, Vice Chairperson 2018/19, 2017/18}{}{}{
  \begin{itemize}
    \item Received the \textit{Most Outstanding EMB Student Branch Chapter Regional Award} for the term 2019/20 (Asia-Pacific region)
    \item Received the \textit{IEEE Darrel Chong Award (Silver Category)} for the term 2019/20
  \end{itemize}
}

\cventry{2016-2017}{Mathematics Society, University of Moratuwa}
{Assistant Secretary 2016/17}{}{}{}

% \section{Other}
% \cvitem{Legal Name}{Laknath Ashwin De Silva Kariyawasam Gonapinuwala Gamage}



%----------------------------------------------------------------------------------------
%	Position of Responsibility SECTION
%----------------------------------------------------------------------------------------

% \section{Position of Responsibility}
% \cventry{2016-2020}{Executive member of IEEE Student Branch}{}{IIT ABC}{}{}
% \cventry{April 1-5, 2019}{Organizer, GIAN Workshop on subjects}{}{IIT ABC}{}{}


% \section{Referees}


% \begin{tabular}{lr}
% % Referee 1
% \begin{minipage}[t]{3in}
% \textbf{Dr. XXXXX XXXXX}\\
% \textit{Associate Professor, Department of} \\
% \textit{Computer Science \& Engineering}\\
% Institute name\\
% \Letter\ \href{mailto:abc@gmail.com}{abc@gmail.com}
% \end{minipage}
% &
% % Referee 2
% \begin{minipage}[t]{3in}
% \textbf{Dr. XXXXX XXXXX}\\
% \textit{Associate Professor, Department of} \\
% \textit{Computer Science \& Engineering}\\
% Institute name\\
% \Telefon\ +(601) 877-6236\\
% \Letter\ \href{mailto:abc@gmail.com}{abc@gmail.com}
% \end{minipage}
% \\
% \\ % Additional newline for spacing.
% % Referee 3
% \begin{minipage}[t]{3in}
% \textbf{Dr. XXXXX XXXXX}\\
% \textit{Associate Professor, Department of} \\
% \textit{Computer Science \& Engineering}\\
% Institute name\\
% \Telefon\ +(601) 877-6236\\
% \Letter\ \href{mailto:abc@gmail.com}{abc@gmail.com}
% \end{minipage}
% &
% \\
% \end{tabular}


\end{document}