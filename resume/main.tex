\documentclass[10pt, letterpaper]{article}

% Packages:
\usepackage[
    ignoreheadfoot, % set margins without considering header and footer
    top=1.5 cm, % seperation between body and page edge from the top
    bottom=1.5 cm, % seperation between body and page edge from the bottom
    left=2 cm, % seperation between body and page edge from the left
    right=2 cm, % seperation between body and page edge from the right
    footskip=1.0 cm, % seperation between body and footer
    % showframe % for debugging 
]{geometry} % for adjusting page geometry
\usepackage{titlesec} % for customizing section titles
\usepackage{tabularx} % for making tables with fixed width columns
\usepackage{array} % tabularx requires this
\usepackage[dvipsnames]{xcolor} % for coloring text
\definecolor{primaryColor}{RGB}{0, 0, 0} % define primary color
\usepackage{enumitem} % for customizing lists
\usepackage{fontawesome5} % for using icons
\usepackage{amsmath} % for math
\usepackage[
    pdftitle={John Doe's CV},
    pdfauthor={John Doe},
    pdfcreator={LaTeX with RenderCV},
    colorlinks=true,
    urlcolor=primaryColor
]{hyperref} % for links, metadata and bookmarks
\usepackage[pscoord]{eso-pic} % for floating text on the page
\usepackage{calc} % for calculating lengths
\usepackage{bookmark} % for bookmarks
\usepackage{lastpage} % for getting the total number of pages
\usepackage{changepage} % for one column entries (adjustwidth environment)
\usepackage{paracol} % for two and three column entries
\usepackage{ifthen} % for conditional statements
\usepackage{needspace} % for avoiding page brake right after the section title
\usepackage{iftex} % check if engine is pdflatex, xetex or luatex
\usepackage{ragged2e}

% Ensure that generate pdf is machine readable/ATS parsable:
\ifPDFTeX
    \input{glyphtounicode}
    \pdfgentounicode=1
    \usepackage[T1]{fontenc}
    \usepackage[utf8]{inputenc}
    \usepackage{lmodern}
\fi

\usepackage{charter}

% Some settings:
\raggedright
\AtBeginEnvironment{adjustwidth}{\partopsep0pt} % remove space before adjustwidth environment
\pagestyle{empty} % no header or footer
\setcounter{secnumdepth}{0} % no section numbering
\setlength{\parindent}{0pt} % no indentation
\setlength{\topskip}{0pt} % no top skip
\setlength{\columnsep}{0.15cm} % set column seperation
\pagenumbering{gobble} % no page numbering

\titleformat{\section}{\needspace{4\baselineskip}\bfseries\large}{}{0pt}{}[\vspace{1pt}\titlerule]

\titlespacing{\section}{
    % left space:
    -1pt
}{
    % top space:
    0.3 cm
}{
    % bottom space:
    0.2 cm
} % section title spacing

\renewcommand\labelitemi{$\vcenter{\hbox{\small$\bullet$}}$} % custom bullet points
\newenvironment{highlights}{
    \begin{itemize}[
        topsep=0.10 cm,
        parsep=0.10 cm,
        partopsep=0pt,
        itemsep=0pt,
        leftmargin=0 cm + 10pt
    ]
}{
    \end{itemize}
} % new environment for highlights


\newenvironment{highlightsforbulletentries}{
    \begin{itemize}[
        topsep=0.10 cm,
        parsep=0.10 cm,
        partopsep=0pt,
        itemsep=0pt,
        leftmargin=10pt
    ]
}{
    \end{itemize}
} % new environment for highlights for bullet entries

\newenvironment{onecolentry}{
    \begin{adjustwidth}{
        0 cm + 0.00001 cm
    }{
        0 cm + 0.00001 cm
    }
}{
    \end{adjustwidth}
} % new environment for one column entries

\newenvironment{twocolentry}[2][]{
    \onecolentry
    \def\secondColumn{#2}
    \setcolumnwidth{\fill, 4.5 cm}
    \begin{paracol}{2}
}{
    \switchcolumn \raggedleft \secondColumn
    \end{paracol}
    \endonecolentry
} % new environment for two column entries

\newenvironment{threecolentry}[3][]{
    \onecolentry
    \def\thirdColumn{#3}
    \setcolumnwidth{, \fill, 4.5 cm}
    \begin{paracol}{3}
    {\raggedright #2} \switchcolumn
}{
    \switchcolumn \raggedleft \thirdColumn
    \end{paracol}
    \endonecolentry
} % new environment for three column entries

\newenvironment{header}{
    \setlength{\topsep}{0pt}\par\kern\topsep\centering\linespread{1.5}
}{
    \par\kern\topsep
} % new environment for the header

\newcommand{\placelastupdatedtext}{% \placetextbox{<horizontal pos>}{<vertical pos>}{<stuff>}
  \AddToShipoutPictureFG*{% Add <stuff> to current page foreground
    \put(
        \LenToUnit{\paperwidth-2 cm-0 cm+0.05cm},
        \LenToUnit{\paperheight-1.0 cm}
    ){\vtop{{\null}\makebox[0pt][c]{
        \small\color{gray}\textit{Last updated in September 2024}\hspace{\widthof{Last updated in September 2024}}
    }}}%
  }%
}%

% save the original href command in a new command:
\let\hrefWithoutArrow\href

% new command for external links:


\begin{document}
    \newcommand{\AND}{\unskip
        \cleaders\copy\ANDbox\hskip\wd\ANDbox
        \ignorespaces
    }
    \newsavebox\ANDbox
    \sbox\ANDbox{$|$}

    \begin{header}
        \fontsize{25 pt}{25 pt}\selectfont Ashwin De Silva

        \vspace{5 pt}

        \normalsize
        \mbox{Clark Hall 317, Baltimore MD 21218}%
        \kern 5.0 pt%
        \AND%
        \kern 5.0 pt%
        \mbox{\hrefWithoutArrow{mailto:youremail@yourdomain.com}{ldesilv2@jhu.edu}}%
        \kern 5.0 pt%
        \AND%
        \kern 5.0 pt%
        \mbox{\hrefWithoutArrow{https://yourwebsite.com/}{https://laknath1996.github.io/}}%
    \end{header}

    \vspace{5 pt - 0.3 cm}

    \section{Summary}
        \begin{onecolentry}
            \begin{justify}
                Ph.D. candidate in Biomedical Engineering with expertise in statistical machine learning and deep learning, focused on developing robust models for out-of-distribution generalization and continual learning. Passionate about bridging artificial and natural intelligence. My work spans applications in large language models, biomedical data, and computer vision.
            \end{justify}
   
            % Highly motivated Ph.D. candidate with a strong background in statistical machine learning and deep learning. My research focuses on out-of-distribution generalization and continual learning, with the aim of bridging the gap between artificial and natural intelligence. I apply these methods in diverse domains, including large language models, computer vision, and biomedical data science.
        \end{onecolentry}
    
    % \section{Quick Guide}

    % \begin{onecolentry}
    %     \begin{highlightsforbulletentries}


    %     \item Each section title is arbitrary and each section contains a list of entries.

    %     \item There are 7 unique entry types: \textit{BulletEntry}, \textit{TextEntry}, \textit{EducationEntry}, \textit{ExperienceEntry}, \textit{NormalEntry}, \textit{PublicationEntry}, and \textit{OneLineEntry}.

    %     \item Select a section title, pick an entry type, and start writing your section!

    %     \item \href{https://docs.rendercv.com/user_guide/}{Here}, you can find a comprehensive user guide for RenderCV.


    %     \end{highlightsforbulletentries}
    % \end{onecolentry}

    \section{Education}



        
        \begin{twocolentry}{
            Aug 2021 – May 2026
        }
            \textbf{Johns Hopkins University}, Ph.D. in Biomedical Engineering\end{twocolentry}

        \vspace{0.10 cm}
        \begin{onecolentry}
            \begin{highlights}
                \item GPA: 3.97/4.00
                \item \textbf{Coursework:} Machine Learning, Optimal Transport, Probability Theory, Statistical Theory, High-Dimensional Approximation, Probabilistic Models of Visual Cortex, Neuroscience and Cognition, Computational Molecular Medicine, Compressed Sensing Sparse Recovery, Entrepreneurial Finance
            \end{highlights}
        \end{onecolentry}

        \vspace{0.20 cm}
        
        \begin{twocolentry}{
            Aug 2021 – May 2024
        }
            \textbf{Johns Hopkins University}, M.S.E in Applied Mathematics and Statistics\end{twocolentry}

        \vspace{0.10 cm}
        \begin{onecolentry}
            \begin{highlights}
                \item GPA: 3.98/4.00
                \item \textbf{Focus Area:} Statistics and Statistical Learning
            \end{highlights}
        \end{onecolentry}

        \vspace{0.20 cm}
        
        \begin{twocolentry}{
            Jan 2016 – Jan 2020
        }
            \textbf{University of Moratuwa}, B.Sc. in Biomedical Engineering \end{twocolentry}

        \vspace{0.10 cm}
        \begin{onecolentry}
            \begin{highlights}
                \item GPA: 4.09/4.20
                \item Class Rank: 1 / 948, Dean's Honors List in every semester
                \item \textbf{Coursework:} Machine Vision, Signals and Systems, Digital Signal Processing, Data Structures and Algorithms, Real Analysis, Calculus, Differential Equations, Linear Algebra
            \end{highlights}
        \end{onecolentry}



    
    \section{Selected Experience}



        
        \begin{twocolentry}{
            Jun 2024 – Sept 2024
        }
            \textbf{Applied Scientist Intern}, Amazon AWS AI -- Santa Clara, CA\end{twocolentry}

        \vspace{0.10 cm}
        \begin{onecolentry}
            \begin{highlights}
                \item Designed, trained, and validated large language models based on State-Space Models (SSMs). These models were used for code generation and lexically constrained text generation.
            \end{highlights}
        \end{onecolentry}


        \vspace{0.2 cm}

        \begin{twocolentry}{
            Jan 2020 – Aug 2021
        }
            \textbf{Junior Lecturer}, University of Moratuwa -- Sri Lanka\end{twocolentry}

        \vspace{0.10 cm}
        \begin{onecolentry}
            \begin{highlights}
                \item Conducted research on deep learning methods for biomedical signal and image analysis, including retinal vascular segmentation, surface EMG-based hand gesture recognition, human pose estimation, and phase unwrapping. This work led to publications in IEEE ICASSP and IEEE SMC.
                \item Taught signals and systems, circuits and systems design, and analysis of physiological systems.
            \end{highlights}
        \end{onecolentry}

         \vspace{0.2 cm}

        \begin{twocolentry}{
            Jun 2018 – Dec 2018
        }
            \textbf{Research Assistant}, Florey Institute of Neuroscience and Mental Health -- Australia\end{twocolentry}

        \vspace{0.10 cm}
        \begin{onecolentry}
            \begin{highlights}
                \item Designed and implemented software tools for electrophysiological signal analysis, enabling large-scale data processing and biomarker extraction. This work was used to support genetic epilepsy research.
            \end{highlights}
        \end{onecolentry}



    
    \section{Selected Publications}

        \begin{onecolentry}
            \begin{highlights}
                 \item \textbf{Ashwin De Silva}, Rahul Ramesh, Rubing Yang, Siyu Yu, Joshua T. Vogelstein, and Pratik Chaudhari. Prospective learning: Learning for a dynamic future. \textit{Neural Information Processing Systems (NeurIPS)}, 2024.
                 \item \textbf{Ashwin De Silva}, Rahul Ramesh, Carey E. Priebe, Pratik Chaudhari, and Joshua T. Vogelstein. The value of out-of-distribution data. \textit{International Conference on Machine Learning (ICML)}, 2023.
            \end{highlights}
        \end{onecolentry}


    \section{Selected Achievements \& Recognition}

    \begin{onecolentry}
        \begin{highlights}
            \item Member, Alpha Eta Mu Beta (AEMB) - National Biomedical Engineering Honor Society (2025)
            \item Mathematical Institute for Data Science (MINDS) Fellowship, Johns Hopkins University (2024)
            \item Johns Hopkins School of Medicine Student Spotlight (2023)
            \item Best Paper Award, ECCV 2022 Workshop on Out-of-distribution Generalization in Computer Vision (2022)
        \end{highlights}
    \end{onecolentry}
        

    
    % \section{Projects}



        
    %     \begin{twocolentry}{
    %         \href{https://github.com/sinaatalay/rendercv}{github.com/name/repo}
    %     }
    %         \textbf{Multi-User Drawing Tool}\end{twocolentry}

    %     \vspace{0.10 cm}
    %     \begin{onecolentry}
    %         \begin{highlights}
    %             \item Developed an electronic classroom where multiple users can simultaneously view and draw on a "chalkboard" with each person's edits synchronized
    %             \item Tools Used: C++, MFC
    %         \end{highlights}
    %     \end{onecolentry}


    %     \vspace{0.2 cm}

    %     \begin{twocolentry}{
    %         \href{https://github.com/sinaatalay/rendercv}{github.com/name/repo}
    %     }
    %         \textbf{Synchronized Desktop Calendar}\end{twocolentry}

    %     \vspace{0.10 cm}
    %     \begin{onecolentry}
    %         \begin{highlights}
    %             \item Developed a desktop calendar with globally shared and synchronized calendars, allowing users to schedule meetings with other users
    %             \item Tools Used: C\#, .NET, SQL, XML
    %         \end{highlights}
    %     \end{onecolentry}


    %     \vspace{0.2 cm}

    %     \begin{twocolentry}{
    %         2002
    %     }
    %         \textbf{Custom Operating System}\end{twocolentry}

    %     \vspace{0.10 cm}
    %     \begin{onecolentry}
    %         \begin{highlights}
    %             \item Built a UNIX-style OS with a scheduler, file system, text editor, and calculator
    %             \item Tools Used: C
    %         \end{highlights}
    %     \end{onecolentry}



    
    % \section{Technologies}



        
    %     \begin{onecolentry}
    %         \textbf{Languages:} C++, C, Java, Objective-C, C\#, SQL, JavaScript
    %     \end{onecolentry}

    %     \vspace{0.2 cm}

    %     \begin{onecolentry}
    %         \textbf{Technologies:} .NET, Microsoft SQL Server, XCode, Interface Builder
    %     \end{onecolentry}


    

\end{document}